\begin{MyRecipe}{Panna Cotta}{\Calc{4}{\x} Portionen}{\SI{30}{\minuteprime}+ \SI{5}{\hour}}
	\ingredient[\Calc{1}{\x}]{} {Vanilleschote}
	\ingredient[\Calc{4}{\x}]{Blatt} {Gelatine}
	\ingredient[\Calc{0.4}{\x}]{\si{\kilogram}} {Sahne}
	\ingredient[\Calc{120}{\x}]{\si{\gram}} {Zucker}
	Für die Panna Cotta die Vanilleschote einritzen und mit dem Messerrücken auskratzen. Gelatine in einer Schale mit kaltem Wasser ca. \SI{5}{\minuteprime} einweichen. Sahne und Zucker mit dem Vanillemark in einem Topf kurz aufkochen, bis sich der Zucker aufgelöst hat.\par\bigskip
	Topf vom Herd nehmen. Die eingeweichten Gelatine leicht ausdrücken, in den Topf mit der noch heißen (nicht mehr kochenden) Flüssigkeit geben. Kurz durchrühren, bis sich die Gelatine aufgelöst hat. Schälchen (à ca. 125 ml) bereitstellen.\par\bigskip
	Die flüssige Panna Cotta hineingießen und die Schälchen ca. \SIrange{10}{15}{\minuteprime} abkühlen lassen. Das Dessert anschließend mindestens 5 Stunden, am besten über Nacht, kühl stellen.\par\bigskip
\end{MyRecipe}