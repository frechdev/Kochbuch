\begin{MyRecipe}{Wiener Saftgulasch}{4 Personen}{20'' + 180''}
	
		\ingredient[2]{EL}{Butterschmalz}
		\ingredient[1]{kg}{Rindergulasch (Wade)}
		\ingredient[800]{g}{Zwiebeln}
		\ingredient[2]{EL}{Tomatenmark}
		\ingredient[2]{}{Knoblauchzehen}
		\ingredient[1]{EL}{milder Essig}
		\ingredient[1]{Schuss}{Rotwein}
		\ingredient[1]{EL}{Majoran}
		\ingredient[1]{EL}{Kümmel}
		\ingredient[2]{EL}{Paprika, edelsüß}
		\ingredient[2]{EL}{Paprika, rosenscharf}
		\ingredient[]{}{Salz und Pfeffer}
		
		
		

\begin{enumerate}
	\item Zwiebel grob hacken und im Butterschmalz ca. 10'' bei niedriger Temperatur glasig (nicht zu dunkel) dünsten. In den letzten 2 Minuten den Knoblauch zugeben.

	\item Tomatenmark zugeben und kurz mitrösten. Dann mit Wein (hilfsweise mit Wasser) ablöschen. Den Wein ein bis zwei Minuten reduzieren lassen und die Gewürze und Essig hinzugeben und eventuell etwas Brühe (hilfsweise mit Wasser) dazugießen.
	\item Das Ganze 1/2 Stunde köcheln lassen, damit sich das Paprika-Aroma voll entfalten kann.
	\item Dann das Fleisch dazu. Das Fleisch soll wirklich nur gerade so mit Flüssigkeit bedeckt sein und sollte noch zu sehen sein. Gegebenenfalls etwas Brühe zugeben.
	\item Das Gulasch 2-3 Stunden auf niedriger Hitze vor sich hinköcheln lassen, bis sich eine dicke Soße bildet und das Fleisch schön zart ist.\\

	Alternativ kann man das Gulasch im Bräter zubereiten, den Deckel auflegen und zugedeckt bei 160 Grad 2 bis 2,5 Stunden in den Ofen geben! Mit Salz und Pfeffer abschmecken und servieren. 
\end{enumerate}

		
	\end{MyRecipe}