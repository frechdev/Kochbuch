\begin{MyRecipe}{Marmor Kuchen}{\Calc{1}{\x} Kuchen}{\SI{30}{\minuteprime} + \SI{60}{\minuteprime}}
	
	\ingredient[\Calc{0.2}{\x}]{\si{\kilogram}} {Butter, weich}
	\ingredient[\Calc{0.16}{\x}]{\si{\kilogram}} {Zucker}
	\ingredient[\Calc{1}{\x}]{\si{\Paeckchen}} {Bourbon-Vanillezucker}
	\ingredient[\Calc{1}{\x}]{\si{\Messerspitze}} {Zitronenabrieb}
	\ingredient[\Calc{2}{\x}]{\si{\Essloeffel}} {Rum}
	
	\Step{Grundteig}
	Die wirklich weiche Butter mit Zucker und Vanillezucker sowie etwas Zitronenabrieb und dem Rum cremig rühren.
	
	\ingredient[\Calc{6}{\x}]{} {Eier}
	\ingredient[\Calc{1}{\x}]{\si{\Messerspitze}} {Salz}
	\ingredient[\Calc{0.12}{\x}]{\si{\kilogram}} {Zucker}
	\Step{Eier hinzugeben}
	Eier trennen. Die Eidotter einzeln nacheinander in die Butter-Zucker-Masse rühren. Die Eiweiße mit dem Salz halbfest schlagen und mit den restlichen 120 g Zucker zu Schnee schlagen.
	
	\ingredient[\Calc{0.28}{\x}]{\si{\kilogram}} {Mehl}
	\ingredient[\Calc{0.5}{\x}]{\si{\Paeckchen}} {Backpulver}
	\ingredient[\Calc{0.1}{\x}]{\si{\liter}} {Milch, lauwarme}
	\Step{Teig abschließen}
	Mehl abwiegen, mit Backpulver mischen. Milch leicht anwärmen. Abwechselnd (etwa in 3 - 4 Schritten) zuerst etwas Mehl auf die Butter-Zucker-Eidotter-Masse sieben (geht gut mit einem großen Haarsieb), dann etwas lauwarme Milch dazugießen und eine Portion Eischnee darauf geben. Mit einem Holzlöffel, der idealerweise in der Mitte ein Loch hat, alles unterheben (das ist sehr wichtig - kein Rührgerät einsetzen).
	
	\ingredient[\Calc{20}{\x}]{\si{\gram}} {Kakao}
	\Step{Teig in Form geben}
	Wenn alles untergehoben und sorgfältig vermischt ist (geht relativ leicht, da der Teig locker bleibt), die Kuchenform gut mit Butter ausstreichen und mit Mehl bestäuben. Gut die Hälfte des Teigs in die Form füllen.
	
	Den restlichen Teig mit dem Kakaopulver dunkel färben. Dazu das Kakaopulver in die verbliebene Teigmasse sieben, um Klümpchen zu verhindern (Instantpulver würde ich nicht nehmen, da es sich im Teig nicht richtig auflöst). Klassisch ist einfach das dunkle Kakaopulver, wie es von unseren Großmüttern früher verwendet wurde.
	
	Den dunklen Teig auf die helle Teigmasse in die Form füllen und mit einer Gabel spiralförmig unterziehen. Beide Teigsorten vermischen sich dadurch zum Marmormuster (mit einer großen Fleisch- oder Grillgabel, die man leicht schräg hält, geht es besonders leicht und man erzielt dadurch ein perfektes Muster).
	
	\Step{Backen}
	Im vorgeheizten Backofen bei 150 - 160 °C Umluft (180 °C Ober-/Unterhitze) ca. 60 Minuten backen. Er soll jedenfalls keine zu harte Kruste kriegen und eher heller backen, damit er auf keinen Fall trocken wird. Man sollte natürlich die Stäbchenprobe machen.
	
	Den Kuchen in der Form leicht auskühlen lassen und dann auf ein Kuchengitter stürzen.
	
	\ingredient[\Calc{0.4}{\x}]{\si{\kilogram}} {Schokoladen Kouvertüre}
	\Step{Überziehen}
	Sobald Kuchen einigermaßen abgekühlt ist, mit Schokoladen Kouvertüre überziehen und trocknen lassen.
	
\end{MyRecipe}