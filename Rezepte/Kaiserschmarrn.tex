\begin{MyRecipe}{Kaiserschmarrn}{\Calc{2}{\x} Personen}{\SI{15}{\minuteprime} + \SI{30}{\minuteprime} + \SI{15}{\minuteprime}}
	
	\ingredient[\Calc{60}{\x}]{\si{\gram}} {Rosinen}
	\ingredient[\Calc{5}{\x}]{EL} {Rum/Wasser}
	\ingredient[\Calc{30}{\x}]{\si{\gram}} {Butter}
	\Step{Vorbereitung}
	Rosinen gut waschen, vom Strunk befreien und in Rum bzw. Wasser einlegen.
	
	In einer großen Pfanne (in der später auch der Kaiserschmarrn gebacken wird) auf niedriger Temperatur die Butter zum schmelzen bringen.\par\bigskip
	
	\ingredient[\Calc{4}{\x}]{} {Eier}
	\ingredient[\Calc{1}{\x}]{Pck} {Vanillezucker}
	\ingredient[\Calc{1}{\x}]{Pr} {Salz}
	\ingredient[\Calc{1}{\x}]{geh. EL} {Zucker}
	\ingredient[\Calc{0.16}{\x}]{\si{\kilogram}} {Mehl}
	\ingredient[\Calc{0.33}{\x}]{\si{\kilogram}} {Milch}
	\Step{Teig anrühren}
	Eier trennen. Die Eigelb in eine große Schüssel und die Eiweiß zum Eischnee machen in ein hohes Gefäß für später geben.
	
	Zum Eigelb Vanillezucker, Salz und Zucker geben und schaumig rühren.
	
	Nach und nach das Mehl (gesiebt) und die Milch im wechsel und in möglichst kleinen Portionen einrühren.
	
	Anschließend die zerlassene Butter langsam und unter ständigem Rühren einfließen lassen.
	
	Sobald alles verrührt ist, den Teig abdecken und \SI{30}{\minuteprime} stehen lassen.\par\bigskip
	
	\ingredient[]{etw.} {Butterschmalz}
	\Step{Backen}
	Unmittelbar vor dem ausbacken das Eiweiß zu einem festen Schnee aufschlagen und mit einem Löffel langsam und vorsichtig, aber gründlich unter die Teigmasse heben.
	
	In der Pfanne einen guten Schlag (ca. \SI{20}{\gram}/Pfanne) Butterschmalz erlassen und dann den Teig ca. \SI{1}{\centi\meter} hoch eingießen. Die Hitze etwas reduzieren und den Teig goldgelb anbacken lassen. Währenddessen die eingelegten Rosinen großzügig über dem Teig verteilen.
	
	Die Masse mit einem Pfannenwender zur Mitte hin vierteln, die einzelnen Stücke wenden und wieder goldgelb anbacken lassen.
	
	Achtung, hier könnts bissle hektisch werden:
	Die Masse in mundgerechte Stücke zerteilen, mit etwas Zucker bestreuen und karamellisieren lassen. Der Zucker darf leicht braun werden sollte aber in keinem Fall verbrennen. Sobald der Zucker ausreichend karamellisiert ist, mit dem Rum der Rosinen großzügig bespritzen und die Pfanne vorsichtig schwenken.\par\bigskip
	
	\ingredient[\Calc{0.2}{\x}]{\si{\kilogram}} {Apfelmus}
	\ingredient[]{etw.} {Puderzucker}
	\Step{Servieren}
	Den Kaiserschmarrn auf einen großen Teller geben, großzügig mit Puderzucker bestreuen und dazu Apfelmus servieren.\par\bigskip
	
	\Step{Tipp}
	Für größere Mengen können die einzelnen Pfannen-Portionen auch erst mal im Backofen bei \SI{80}{\degreeCelsius} Ober-/Unterhitze warmgehalten werden.

	
	

	
	
	
	
\end{MyRecipe}