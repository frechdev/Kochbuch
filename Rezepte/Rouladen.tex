\begin{MyRecipe}{Rouladen}{\Calc{6}{\x} Personen}{30'' + 120''}

\ingredient[\Calc{2}{\x}]{kg}{Rinderrouladen aus der Oberschale, mit schöner Marmorierung}
\ingredient[\Calc{0.2}{\x}]{kg}{fetter Speck}
\ingredient[\Calc{2}{\x}]{}{saure Gurken}
\ingredient[\Calc{2}{\x}]{}{Zwiebeln}

\textbf{Vorbereitung:}\\
Rindfleisch in dünne aber feste, möglichst große Scheiben schneiden (lassen).\\
Speck und Gurken in schmale Streifen, Zwiebeln in dünne Halbringe schneiden.\par\bigskip

\ingredient[\Calc{0.2}{\x}]{kg}{Senf, mittelscharf}
\ingredient[\Calc{1}{\x}]{EL}{Butter}
\ingredient[\Calc{1}{\x}]{EL}{Pflanzenöl}
\ingredient[]{}{Spieße / Bindfaden}
\ingredient[]{}{Salz}
\ingredient[]{}{Pfeffer}

\textbf{Zubereitung:}\\
Fleischscheiben auf Arbeitsfläche auslegen. Mit Senf bestreichen, mit Salz und Pfeffer kräftig Würzen. Mit Gurken, Speck und Zwiebeln belegen. Vorsichtig zusammenrollen und mit Spießen oder Bindfaden fixieren.\par
Nach und nach die Rouladen in Bräter mit Butter-Öl-Gemisch anbraten, raus nehmen und mit etwas Wasser den Fond ablösen, bis alle Rouladen angebraten sind.\par
Die angebratenen Rouladen zusammen mit dem abgelösten Fond in die Kachel geben und heißem Wasser auffüllen, bis die Rouladen sollten fast bedeckt sind. \SI{90}{\min} im Backofen bei \SI{200}{\degreeCelsius} schmoren. Dabei von Zeit zu Zeit die oberste Schicht wenden, evtl. etwas Wasser zugießen und den Fond vom Bräterrand lösen.\par\bigskip

\ingredient[\Calc{2}{\x}]{kg}{Sauerrahm}
\ingredient[\Calc{1}{\x}]{EL}{Mehl}
\ingredient[\Calc{50}{\x}]{g}{süße Sahne}

\textbf{Soße:}\\
Rouladen vorsichtig aus dem Bräter holen und beiseite legen. Sauerrahm mit Mehl verquirlen und unterrühren, kurz aufkochen und mit Sahne abschmecken. Rouladen wieder zurück in die Soße einlegen.\par\bigskip


\textbf{Das passt dazu:}\\
Bandnudeln und Wirsing oder Rosenkohl

\end{MyRecipe}