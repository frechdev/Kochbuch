\begin{MyRecipe}{Maultaschen}{\Calc{65}{\x} Maultaschen}{\SI{140}{\minuteprime} }

	
	


\ingredient[\Calc{0.25}{\x}]{\si{\kilogram}} {Rauschfleisch}
\ingredient[\Calc{1}{\x}]{} {kl. Lauch}
\ingredient[\Calc{1.5}{\x}]{} {Frühlingszwiebeln}
\ingredient[\Calc{0.3}{\x}]{\si{\kilogram}} {Spinat}
\ingredient[\Calc{2}{\x}]{} {Zwiebeln}
\ingredient[\Calc{3}{\x}]{} {Brötchen, trocken}
\ingredient[\Calc{1}{\x}]{\si{\Bund}} {Petersilie}
\ingredient[\Calc{1}{\x}]{Paar} {Landjäger}

\Step{Vorbereitung}
Rauchfleisch zusammen mit dem Lauch, Zwiebeln und Spinat anbraten.	Brötchen würfeln. Perterling und Landjäger hacken.\par\bigskip

\ingredient[\Calc{1}{\x}]{\si{\kilogram}} {Hackfleisch, gemischt}
\ingredient[\Calc{0.5}{\x}]{\si{\kilogram}} {Brät}
\ingredient[\Calc{6.5}{\x}]{} {Eier}
\ingredient[]{} {Salz}
\ingredient[]{} {Pfeffer}
\ingredient[]{} {Muskat}

\Step{Teig ansetzen}
Angebratenes leicht abkühlen lassen. Dann Zusammen mit dem restlichen Vorbereiteten, dem Hackfleisch und Brät und den Eiern gut durchmischen. Wenn zu trocken, etwas Wasser zugeben. Mit Salz, Pfeffer und Muskat würzen und abschmecken.\par\bigskip

\ingredient[\Calc{1.2}{\x}]{\si{\kilogram}} {Nudelteig}
	
\Step{Taschen formen}
Teig in ca. \SI{12}{\centi\meter} große Quadrate schneiden, mit einem Pinsel befeuchten und mit ca. \SI{40}{\gram} Füllung belegen. Dann den Teig von beiden Seiten einschlagen und von der Mitte aus zu Taschen formen. Hierfür vorsichtig die Luft ausstreichen und die Ränder gut andrücken. Eine Teiglänge an Maultaschen vorbereiten und dann ins siedende Salzwasser geben. Nochmals aufkochen lassen und dann vom Herd nehmen. Insgesammt ca. \SI{10}{\minuteprime} im Topf ziehen lassen. Auf einem Gitterrost kurz abkühlen und dann auf ein mit Backpapier ausgelegtes Backblech verteilen.





	
\end{MyRecipe}