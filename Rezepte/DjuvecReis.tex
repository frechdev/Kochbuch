\begin{MyRecipe}{Djuvec-Reis}{\Calc{4}{\x} Personen als Beilage}{}
	\ingredient[\Calc{1}{\x}]{} {kl. Paprika}
	\ingredient[\Calc{1}{\x}]{} {kl. Zwiebel}
	\ingredient[\Calc{50}{\x}]{\si{\gram}} {Ajvar}
	\ingredient[\Calc{0.2}{\x}]{\si{\liter}} {Wasser}
	\ingredient[]{} {Salz}
	\ingredient[]{} {Zucker}
	\Step{Vorbereiten}
	Zwiebeln und Paprika würfeln.
	
	Ajvar in Wasser auflösen und mit Salz und etw. Zucker abschmecken.
	
	\ingredient[\Calc{100}{\x}]{\si{\gram}} {Reis}
	\ingredient[\Calc{50}{\x}]{\si{\gram}} {TK-Erbsen}
	\ingredient[]{} {Olivenöl}
	\Step{Reis ansetzen}
	Zwiebeln und Paprika mit Olivenöl in Topf glasig braten. Dann Reis hinzufügen und kurz mit andünsten und mit Wasser-Ajvar-Mischung ablöschen. Einmal Kräftig aufkochen lassen, vom Herd nehmen und dann mit geschlossenem Deckel circa \SI{20}{\minute} ziehen lassen.
	
	Erbsen in den fertigen Reis unter rühren.
	

	
	
	
\end{MyRecipe}