\begin{MyRecipe}{Grießbrei}{\Calc{4}{\x} Personen}{\SI{15}{\minuteprime} + \SI{10}{\minuteprime}}
	
	\ingredient[\Calc{0.8}{\x}]{\si{\liter}} {Milch}
	\ingredient[\Calc{80}{\x}]{\si{\gram}} {Weichweizengrieß}
	\ingredient[\Calc{2}{\x}]{\si{\Essloeffel}} {Zucker}
	\ingredient[\Calc{1}{\x}]{} {Vanilleschote}
	\ingredient[\Calc{1}{\x}]{\si{\Prise}} {Salz}
	\ingredient[\Calc{25}{\x}]{\si{\gram}} {Butter}
	\Step{Ansetzen}
	Die Milch in einen Topf mit möglichst großer Bodenfläche geben.
	
	Vanilleschote aufschlitzen und das Mark der Vanilleschote sowie die aufgeschlitzte Vanilleschote selbst zusammen mit dem Zucker, der Butter und einer Prise Salz hinzugeben und zum Kochen bringen.
	
	Anschließend den Weizengrieß unter ständigem Rühren mit einem Schneebesen einrieseln lassen und nochmals aufkochen. Dann den Topf vom Herd nehmen und den Grieß zugedeckt \SI{5}{\minuteprime} ziehen lassen.\par\bigskip
	
	\ingredient[\Calc{1}{\x}]{} {Eier}
	\Step{Ei einrühren (Optional)}
	In der Zwischenzeit das Eigelb vom Eiweiß trennen. Das Eiweiß zu steifem Schnee schlagen. Das Eigelb gründlich in den Grießbrei rühren.
	
	Zum Schluss den Eischnee vorsichtig unter den fertigen Grießbrei heben.\par\bigskip
	
	\ingredient[]{} {Zimt}
	\ingredient[]{} {Zucker}
	\ingredient[]{} {Kompott}
	\ingredient[]{} {eingemachtes Obst}
	\Step{Servieren}
	Serviert werden kann der Grießbrei mit einem beliebigen Kompott z.B. aus Apfel oder Quitte, Zimt \& Zucker, eingemachtem Obst oder allem zusammen.
	
\end{MyRecipe}