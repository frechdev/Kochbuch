\begin{MyRecipe}{Lasagne}{\Calc{4}{\x} Personen}{}
	\ingredient[\Calc{1}{\x}]{} {gr. Möhre}
	\ingredient[\Calc{0.5}{\x}]{} {kl. Sellerie}
	\ingredient[\Calc{1}{\x}]{} {gr. Zwiebel}
	\ingredient[\Calc{2}{\x}]{} {Knoblauchzehen}
	\Step{Vorbereiten}
	Gemüse ein kleine Würfel schneiden.\par\bigskip
	
	
	\ingredient[\Calc{0.6}{\x}]{\si{\kilogram}} {gem. Hackfleisch}
	\ingredient[\Calc{0.2}{\x}]{\si{\liter}} {Weißwein}
	\ingredient[]{} {Bratöl}
	
	\Step{Braten}
	Gemüsewürfel in Öl anbraten. aus der Pfanne nehmen und mit Weißwein ablöschen. Dann etwa die Hälfte des Hackfleischs ebenfalls in Öl krümelig braten, aus der Pfanne zum Gemüse nehmen und mit Weißwein ablöschen. Zuletzt die zweite Hälfte des Hackfleischs krümelig braten, aus der Pfanne zum Gemüse nehmen und mit Weißwein ablöschen.\par\bigskip
	

	\ingredient[\Calc{0.5}{\x}]{\si{\liter}} {Gemüsebrühe}
	\ingredient[\Calc{0.4}{\x}]{\si{\kilogram}} {Dosentomaten}
	\ingredient[\Calc{2}{\x}]{EL} {Tomatenmark}
	\ingredient[]{} {Salz}
	\ingredient[]{} {Pfeffer}
	\ingredient[]{} {Zucker}
	
	\Step{Soße ansetzen}
	Tomatenmark in heißer Pfanne angehen lassen und immer wieder mit etwas Wasser ablöschen. Dann die Gemüse-Fleisch-Mischung zusammen mit den Dosentomaten auf die Soße geben. Mit Salz, Pfeffer und einer Prise Zucker würzen und bei geschlossenem Deckel sämig köcheln lassen.\par\bigskip
	
	\ingredient[\Calc{2}{\x}]{EL} {Butter}
	\ingredient[\Calc{3}{\x}]{gestr. EL} {Mehl}
	\ingredient[\Calc{0.5}{\x}]{\si{\liter}} {Milch}
	\ingredient[]{} {Salz}
	\ingredient[]{} {Pfeffer}
	\ingredient[]{} {Muskat}
	\Step{Béchamelsauce}
	Butter in Topf schmelzen lassen. Dann das Mehl mit einem Schneebesen unter Rühren und kurz angehen lassen. Dann nach und nach die Milch langsam bei ständigem Rühren hinzugeben. Mit Salz, Pfeffer und Muskat abschmecken. Bei möglichst ständigem Rühren einmal aufkochen lassen und vom Herd nehmen.
	
	\ingredient[\Calc{12}{\x}]{} {Lasagneblätter}
	\ingredient[\Calc{0.3}{\x}]{\si{\kilogram}} {Gouda}
	\ingredient[\Calc{0.1}{\x}]{\si{\kilogram}} {Parmesan}
	\Step{Schichten}
	Mit etwas Butter die Auflaufform einfetten. Dann schichten:\par
	Lasagneblätter --- Sauce --- Béchamelsauce --- Käse\par
	Wichtig hierbei ist darauf zu achten, dass ganz oben Béchamelsauce mit Käse liegt.\par\bigskip

	\Step{Backen}
	Bei \SI{180}{\degreeCelsius} Ober-/Unterhitze mind. \SI{20}{\minuteprime} backen, bis der Käse lecker aussieht.
	
	
	
\end{MyRecipe}